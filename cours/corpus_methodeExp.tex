\documentclass[12pt,aspectratio=43,dvipsnames,table]{beamer}
\usepackage[T1]{fontenc}
\usepackage[utf8]{inputenc}
\usepackage[francais]{babel}
\usepackage{url}
\usepackage{tikz}

\usetheme{default}
\useinnertheme{default}
\useoutertheme{default}
\usefonttheme{serif}

\usecolortheme[named=RoyalPurple]{structure}

% Marges autour des slides
\setbeamersize{text margin left=5mm, text margin right=5mm}

% Suppression des symboles de navigation
\setbeamertemplate{navigation symbols}{}

% Définition du pied de page
\setbeamertemplate{footline} {
  \hfill{
    \insertshortdate ~- %
    page \insertframenumber~sur~\inserttotalframenumber %
    %\hspace*{0.1cm}
  } \vspace*{0.05cm}
}

% Définition des espacements des listes
\setlength{\leftmargini}{0.6cm}
\setlength{\leftmarginii}{0.4cm}
\setlength{\leftmarginiii}{0.4cm}

\title{Corpus et méthode expérimentale}
\subtitle{module X9IT080}
\author{Florian Boudin}
\institute{Département informatique, Université de Nantes}
\date[30 juillet 2013 / Rév.~1]{Révision~1 du 30 juillet 2012}

% Pour ajouter le plan au début de chaque section
\AtBeginSection[]
{
\begin{frame}
\frametitle{Plan}
\tableofcontents[sectionstyle=show/shaded,subsectionstyle=hide,subsubsectionstyle=hide]
\end{frame}
}

% Pour pouvoir barrer du texte avec des pauses
\usepackage{ulem} % pour le texte barré sout{}
\renewcommand<>{\sout}[1]{\only#2{\beameroriginal{\sout}{#1}}\invisible#2{#1}}

\begin{document}

%-B--------------------------------------------------------------------------B-%
% frame titre
\frame[plain]{\titlepage}
%-E--------------------------------------------------------------------------E-%

%-B--------------------------------------------------------------------------B-%
\begin{frame}{Présentation du module}
\begin{itemize} \itemsep10pt
    \item Notions abordées dans ce module
	\begin{itemize}
		\item ...
	\end{itemize}
	\item Volume horaire~: 15 séances (20h)
	\begin{itemize}
		\item 6 CM (8h) - 6 TD (8h) - 3 TP (4h)
	\end{itemize}
	\item Notation
\end{itemize}

\end{frame}
%-E--------------------------------------------------------------------------E-%


% Objectifs pédagogiques :
% Les objectifs de cette UE sont doubles : 
% 1. Aborder la notion de corpus, sa constitution, son analyse et les normes linguistiques d'annotation
% 2. Mettre en place une méthode expérimentale, savoir analyser les résultats
% Prérequis :
% Introduction au traitement automatique de la langue naturelle (X7IT010)
% Structure et modélisation en langue naturelle (X8IT010)
% Programme - Contenu de l'UE :
% Définition d'un corpus électronique, les différents types de corpus : monolingues, bilingues, etc.
% Contraintes de constitution d'un corpus, les méthodes de moissonnage à partir du web, les encodages, les méta-données, la caractérisation des textes
% Les normes de caractérisation et d'annotation : CES, TEI, DUBLIN CORE, LGRAF
% Mise en place d'une démarche expérimentale : paramètres d'évaluation (quantité de données, découpage entraînement/dev/test, etc.), analyse des résultats, tests de significativité, tests de correlation
% Compétences acquises :
% Cette UE apporte un ensemble de connaissances sur la notion de corpus, qui représente le matériau de base de nombreuses applications de TALN, et sur la mise en place d'une démarche expérimentale rigoureuse.


%==============================================================================%
\section{La notion de corpus}
%==============================================================================%

%-B--------------------------------------------------------------------------B-%
%-B--------------------------------------------------------------------------B-%
\begin{frame}[allowframebreaks]
\frametitle{Définitions}

\begin{block}{Corpus}

Un \textbf{corpus} est un ensemble de documents (textes, images, vidéos, etc.) 
regroupés dans une optique précise.\\[0.5em]

Dans le cadre de ce cours~: \alert{\fbox{corpus $\to$ corpus de textes}}\\[0.5em]

Plusieurs caractéristiques sont à prendre en compte pour la création d'un corpus
bien formé~:

	\begin{itemize}
		\item la taille;
		\item le langage;
		\item le temps couvert par les textes du corpus;
		\item le registre de langage.
	\end{itemize}

\end{block}

Source~: \url{http://fr.wikipedia.org/wiki/Corpus}

\framebreak

\begin{block}{Taille}

Le corpus doit évidemment atteindre une taille critique pour permettre des 
traitements statistiques fiables.

    \begin{itemize}
        \item Si un corpus est utilisé pour construire des modèles de langue, 
              quelle doit être sa taille minimum~? 1M mots, 10M mots, etc.
    \end{itemize}

\end{block}

\begin{block}{Langage}

Un corpus \textbf{monolingue} bien formé doit nécessairement couvrir une seule 
langue, et une seule déclinaison de cette langue (e.g.~français de France et 
français du Québec).

\end{block}

\framebreak

\begin{block}{Période couverte}

Le temps joue un rôle important dans l'évolution du langage~: le français parlé 
aujourd'hui ne ressemble pas au français parlé il y a 200 ans ni, de façon plus 
subtile, au français parlé il y a 10 ans, à cause notamment des néologismes.

\end{block}


\begin{block}{Registre de langage}

Un corpus construit à partir de textes scientifiques ne peut être utilisé pour 
extraire des informations sur les textes vulgarisés, et un corpus mélangeant des
textes scientifiques et vulgarisés ne permettra pas de tirer de conclusion sur 
ces deux registres.

\end{block}

\end{frame}
%-E--------------------------------------------------------------------------E-%

%-B--------------------------------------------------------------------------B-%
\begin{frame}
\frametitle{Les différents types de corpora}
\tableofcontents[sectionstyle=show/hide,subsectionstyle=show]
\end{frame}
%-E--------------------------------------------------------------------------E-%


%==============================================================================%
\subsection{Corpus parallèle}
%==============================================================================%


%-B--------------------------------------------------------------------------B-%
\begin{frame}
\frametitle{Corpus parallèle}

Un corpus parallèle est un ensemble de paires de textes tel que, pour une 
paire, un des textes est la traduction de l'autre. \\[0.5em]

Construire un corpus parallèle nécessite un \textbf{alignement} des unités 
textuelles~:

    \begin{itemize}
        \item mettre en correspondance des unités textuelles en langue source
              avec celles de la langue cible.
    \end{itemize}

L'alignement des unités textuelles peut être manuel ou automatique.\\[0.5em]

La granularité de l'alignement (documents, phrases, mots) dépend de 
l'utilisation du corpus~:

    \begin{itemize}
        \item traduction automatique, génération de paraphrases, construction
              dictionnaires bilingues, etc.
    \end{itemize}

\end{frame}
%-E--------------------------------------------------------------------------E-%

%-B--------------------------------------------------------------------------B-%
\begin{frame}
\frametitle{Alignement de phrases}

\begin{columns}[c] 
    \small

    \column{.45\textwidth}

    Il s'agit de l'un des sauropodes les plus connus. \\[0.5em]

    C'était un très grand quadrupède au long cou, avec une longue queue en 
    forme de fouet.\\[0.5em]

    Ses membres antérieurs étaient légèrement plus courts que ses membres 
    postérieurs, ce qui lui donnait une posture horizontale.\\[1.2em]
    ~

    \column{.1\textwidth}

    \begin{tikzpicture}
        \draw [ultra thick, <-] (0,5.5) -- (1,4.5);
        \draw [ultra thick, <-] (0,4) -- (1,4);
        \node at (0.6,4.3) {2-1};
        \draw [ultra thick, <-] (0,2) -- (1,2);
        \node at (0.6,2.3) {1-1};
        \draw [ultra thick,|-] (0.5,0) -- (1,0);
        \node at (0.6,0.4) {0-1};
    \end{tikzpicture}
    \vspace*{1em}

    \column{.45\textwidth}

    One of the best-known sauropods, Diplodocus was a very large long-necked 
    quadrupedal animal, with a long, whip-like tail.\\[1.2em]

    Its forelimbs were slightly shorter than its hind limbs, resulting in a 
    largely horizontal posture.\\[1.2em]

    It is the longest dinosaur known from a complete skeleton.

\end{columns}

\vspace*{1em}

Source~: \url{http://en.wikipedia.org/wiki/Diplodocus}

\end{frame}
%-E--------------------------------------------------------------------------E-%


%-B--------------------------------------------------------------------------B-%
\begin{frame}
\frametitle{Alignement de mots}

Exemple d'alignment simple.

\begin{center}
Je suis Français . \\
\begin{tikzpicture}
    \draw [ultra thick, <-] (0.4,0) -- (0.2,1);
    \draw [ultra thick, <-] (0.9,0) -- (0.8,1);
    \draw [ultra thick, <-] (2,0) -- (2,1);
    \draw [ultra thick, <-] (2.9,0) -- (3.1,1);
\end{tikzpicture}\\[-0.2em]
I am French .
\end{center}

\vspace*{1em}

Exemple d'alignment plus compliqué.

\begin{center}
Je m' appelle Paul . \\
\begin{tikzpicture}
    \draw [ultra thick, |-] (0.4,0.5) -- (0.4,1);
    \draw [ultra thick, <-] (0.6,0) -- (0.9,1);
    \draw [dashed, ultra thick, <-] (1.5,0) -- (2,1);
    \draw [ultra thick, -|] (2.3,0) -- (2.3,0.5);
    \draw [ultra thick, <-] (2.9,0) -- (3.1,1);
    \draw [ultra thick, <-] (3.5,0) -- (3.6,1);
\end{tikzpicture}\\[-0.2em]
My name is Paul .
\end{center}

\end{frame}
%-E--------------------------------------------------------------------------E-%

%-B--------------------------------------------------------------------------B-%
\begin{frame}
\frametitle{Corpora parallèles disponibles}

\begin{itemize}\itemsep10pt
    \item Europarl~\cite{koehn2005europarl}
    \begin{itemize}
        \item Délibérations du Parlement européen disponibles en 21 langues et 
              alignées au niveau de la phrase.
    \end{itemize}
    \item OPUS~\cite{tiedemann2009news}
    \begin{itemize}
        \item Collection de corpora parallèles issus de sources variées~:
              EMEA (European Medicines Agency documents), KDE4 (KDE4 
              localization files), Europarl, OpenSubtitles, etc.
        \item \url{http://opus.lingfil.uu.se/}
    \end{itemize}
    \item Les livres numériques librement disponibles.
    \begin{itemize}
        \item \url{http://www.gutenberg.org/}
    \end{itemize}
    \item Wikipédia~: \alert{\fbox{plus comparable que parallèle}}
\end{itemize}

\end{frame}
%-E--------------------------------------------------------------------------E-%


%==============================================================================%
\subsection{Corpus comparable}
%==============================================================================%


%-B--------------------------------------------------------------------------B-%
\begin{frame}[allowframebreaks]
\frametitle{Corpus comparable}

\begin{itemize}\itemsep10pt
    \item Les corpora parallèles sont très couteux à produire et ils ne sont
          disponibles que dans un nombre de langues/domaines réduit.
    \item Les corpus dits \textbf{comparables} sont largement plus répandus.
    \item D'après Déjean \& Gaussier~\cite{dejean2002nouvelle}~:
    \begin{itemize}
        \item Deux corpus de deux langues $l_1$ et $l_2$ sont dits comparables 
              s'il existe une sous-partie non négligeable du vocabulaire du 
              corpus de langue $l_1$, respectivement $l_2$, dont la traduction 
              se trouve dans le corpus de langue $l_2$, respectivement $l_1$.
    \end{itemize}

    \framebreak

    \item Exemple~: un ensemble d'articles de journaux dans différentes langues,
          traitant d'une même actualité et à la même époque.
    \item Applications~:
    \begin{itemize}
        \item Extraction de phrases parallèles~\cite{smith2010extracting}
        \item Constitution de dictionnaires bilingues~\cite{rapp1999automatic}
    \end{itemize}
    \item Peu (ou pas~?) de corpora comparables disponibles.
    \begin{itemize}
        \item Wikipedia
    \end{itemize}
\end{itemize}

\end{frame}
%-E--------------------------------------------------------------------------E-%


%==============================================================================%
\subsection{La constitution de corpus}
%==============================================================================%


%-B--------------------------------------------------------------------------B-%
\begin{frame}
\frametitle{La constitution de corpus}

\begin{enumerate}\itemsep10pt

    \item Définir les caractéristiques du corpus~:
    \begin{itemize}
        \item Quels sont les phénomènes que l'on souhaite observer~?
        \item Quelle est la tâche que l'on souhaite réaliser~?
        \item Le corpus doit-il pouvoir être distribué~?
    \end{itemize} 

    \item Assembler les unités textuelles (documents, phrases, etc.)
    \begin{itemize}
        \item Est-ce que le processus peut être automatisé~?
        \begin{itemize}
            \item e.g.~moissonnage à partir du web (\textit{web scraping}).
        \end{itemize}
        \item Une sélection manuelle est-elle nécessaire (et possible)~?
        \item[$\to$] Garder en mémoire la méthodologie utilisée (e.g.~README).
    \end{itemize}

    \item[Opt.] Annotation du corpus
    \begin{itemize}
        \item Annoter les unités textuelles (e.g.~\textit{Part-Of-Speech}).
        \item Créer un référenciel pour l'évaluation (e.g.~termes-clés).
        \item[$\to$] Définir des \textit{guidelines} à joindre au corpus.
    \end{itemize}

\end{enumerate}

\end{frame}
%-E--------------------------------------------------------------------------E-%

%-B--------------------------------------------------------------------------B-%
\begin{frame}
\frametitle{Exemple 1}
\framesubtitle{Corpus pour évaluer un système d'extraction de termes-clés}

\begin{itemize} \itemsep10pt
    \item Caractéristiques du corpus
    \begin{itemize}
        \item On souhaite évaluer un système d'extraction de termes-clés et 
              pouvoir distribuer le corpus pour des raisons de comparaison.
        \item[$\to$] Choix de la nature des documents~: \alert{articles}, blogs,
                     tweets, etc.
        \item[$\to$] Langue(s) des documents~: anglais, \alert{français}, etc.
        \item[$\to$] Source(s) des documents~: lemonde.fr, \alert{wikinews}, 
                     etc.
        \item[$\to$] Nombre de documents~: 10, 20, 50, \alert{100}, 1000, etc.
    \end{itemize}

    \item Récupération des documents
    \begin{itemize}
        \item Automatiser le processus à partir d'un \textit{dump} de wikinews
        \item[$\to$] Filtrage (manuel) des documents trop courts
        \item Définir un format pour les documents (XML, HTML, txt, etc.)
    \end{itemize}

    \item Création d'un référentiel pour l'évaluation
    \begin{itemize}
        % \item Recruter des annotateurs pour extraire les termes-clés
        \item Tâche subjective $\to$ plusieurs annotations par documents
    \end{itemize}

\end{itemize}

\end{frame}
%-E--------------------------------------------------------------------------E-%

%-B--------------------------------------------------------------------------B-%
\begin{frame}
\frametitle{Exemple 2}
\framesubtitle{Corpus de phrases analysées en dépendances}

\begin{itemize} \itemsep10pt
    \item Caractéristiques du corpus
    \begin{itemize}
        \item On souhaite créer un corpus composé de phrases analysées en 
              dépendances afin d'étudier des phénomènes linguistiques
              et d'entraîner un outil d'analyse en dépendances.
        \item[$\to$] Choix de la nature des phrases~: \underline{\qquad}
        \item[$\to$] Langue(s) des phrases~: \underline{\qquad}
        \item[$\to$] Source(s) des phrases~: \underline{\qquad}
        \item[$\to$] Nombre de phrases~: \underline{\qquad}
    \end{itemize}
    \item Récupération des phrases
    \begin{itemize}
        \item[$\to$] Récupération automatisée ou manuelle~?
    \end{itemize}
    \item Annotation des phrases
    \begin{itemize}
        \item Recruter des spécialistes~? définir des \textit{guidelines}~? 
    \end{itemize}

\end{itemize}

\end{frame}
%-E--------------------------------------------------------------------------E-%


%==============================================================================%
\subsection{L'annotation de corpus}
%==============================================================================%

%-B--------------------------------------------------------------------------B-%
\begin{frame}
\frametitle{L'annotation de corpus}

\begin{itemize}
    \item Différents niveaux d'annotation
\end{itemize}

\end{frame}
%-E--------------------------------------------------------------------------E-%



%-B--------------------------------------------------------------------------B-%
\begin{frame}
\frametitle{Un point sur l'encodage}

\end{frame}
%-E--------------------------------------------------------------------------E-%


%-B--------------------------------------------------------------------------B-%
\begin{frame}
\frametitle{Un point sur les méta-données}

\end{frame}
%-E--------------------------------------------------------------------------E-%



%-E--------------------------------------------------------------------------E-%


%==============================================================================%
\section{Les normes de caractérisation et d'annotation}
%==============================================================================%


%-B--------------------------------------------------------------------------B-%
\begin{frame}
\frametitle{Les normes de caractérisation et d'annotation}
\begin{itemize}
    \item Partie de Béatrice
\end{itemize}
\end{frame}
%-E--------------------------------------------------------------------------E-%


%==============================================================================%
\section{Démarche expérimentale}
%==============================================================================%

%-B--------------------------------------------------------------------------B-%
% Mise en place d'une démarche expérimentale : paramètres d'évaluation (quantité de données, découpage entraînement/dev/test, etc.), analyse des résultats, tests de significativité, tests de correlation

%-B--------------------------------------------------------------------------B-%
\begin{frame}[allowframebreaks]
\frametitle{Introduction}

\begin{itemize} \itemsep0.8em

    \item \textbf{démarche expérimentale} $\to$ valider une hypothèse par des 
          expériences
    \begin{itemize}
        \item e.g.~expérience du cerf-volant de Benjamin Franklin
    \end{itemize}

    \item exemple
    \begin{itemize}
        \item hypothèse $\to$ segmenter le chinois avec les CRF
        \item expérience $\to$ évaluer la précision sur un corpus
    \end{itemize}    

    \item \textbf{évaluer} un système de TAL consiste à vérifier qu'il produit 
          le résultat pour lequel il a été conçu

    \framebreak

    \item le TAL englobe un grand nombre de tâches, chacune ayant des critères 
          particuliers quant à son évaluation

    \item cette partie est basée sur~\cite{resnik2010}
    \begin{itemize}
        \item \url{http://www.umiacs.umd.edu/~jbg/teaching/CMSC_773_2012/reading/evaluation.pdf}
    \end{itemize}
\end{itemize}

\end{frame}
%-E--------------------------------------------------------------------------E-%

%-B--------------------------------------------------------------------------B-%
\begin{frame}[allowframebreaks]
\frametitle{Exemple}

\begin{itemize} \itemsep0.8em
    \item un exemple pour introduire les idées que nous allons voir en détails 
          par la suite

    \item évaluer un moteur de recherche sémantique
    \begin{itemize}
        \item requêtes analysées et converties en sujet-relation-objet
    \end{itemize}

    \item \textit{when was the light bulb patented by Edison~?}
    \begin{itemize}
        \item[$\to$] $[$Edison, patented, bulb$]$
        \item permet de retrouver le document \og{}\textit{Thomas Edison's 
              patent of the electric light bulb}\fg{}
    \end{itemize}

    \framebreak

    \item comment évaluer le composant d'analyse en sujet-relation-objet~?

    \item[$\to$] conduire une évaluation \textbf{intrinsèque}
    \begin{itemize}
        \item créer un ensemble de questions analysées manuellement
        \item évaluer la performance avec des mesures de P-R-F
         \item comparer deux versions (\textit{formative evaluation})
        \item comparer à d'autres méthodes (\textit{summative evaluation})
    \end{itemize}

    \framebreak

    \item est-ce qu'une analyse précise permet d'améliorer les résultats du 
          moteur de recherche~?

    \item[$\to$] conduire une évaluation \textbf{extrinsèque}
    \begin{itemize}
        \item évaluer l'impact de l'analyse sur la performance du moteur
        \item utiliser une collection de test avec des mesures de P-R-F
    \end{itemize}

    \item[$\to$] conduire évaluation \textit{in situ}
    \begin{itemize}
        \item proposer le système aux utilisateurs et les observer
    \end{itemize}

\end{itemize}

\end{frame}
%-E--------------------------------------------------------------------------E-%

%-B--------------------------------------------------------------------------B-%
\begin{frame}[allowframebreaks]
\frametitle{Concepts fondamentaux}

\begin{itemize} \itemsep0.8em
    \item \textbf{évaluation manuelle}
    \begin{itemize}
        \item demander à des sujets humains d'évaluer un système selon des 
              critères pré-définis $\to$ souvent la meilleure évaluation~!
        \item nombreuses limitations~: coût important, évaluation lente, 
              résultats inconsistants et non reproductibles
    \end{itemize}

    \item \textbf{évaluation automatique}
    \begin{itemize}
        \item création d'un \textit{gold standard}, \textit{ground truth}, etc.
        \item nécessite une mesure qui \og{}simule\fg{} une évaluation manuelle
        \item correlation entre les évaluations manuelles et automatiques
    \end{itemize}

    \framebreak

    \item \textbf{évaluation intrinsèque}
    \begin{itemize}
        \item la sortie du système est évaluée directement par rapport à des 
              critères pré-définis
    \end{itemize}

    \item \textbf{évaluation extrinsèque}
    \begin{itemize}
        \item la sortie du système est évaluée à travers son impact sur une 
              tâche externe
    \end{itemize}

    \item exemple du résumé automatique
    \begin{itemize}
        \item évaluation intrinsèque $\to$ ROUGE ou évaluation manuelle
        \item évaluation extrinsèque $\to$ les résumés sont-ils utiles pour 
              remplacer les snippets d'un moteur de RI~?
    \end{itemize}

    \framebreak

    \item \textbf{accord inter-annotateurs}
    \begin{itemize}
        \item en TAL, l'évaluation se résume à annoter du texte
        \item comparer la performance de plusieurs annotateurs
        \item un accord inter-annotateurs faible
        \begin{itemize}
            \item[$\to$] tâche trop difficile ou mal définie
        \end{itemize}
        \item le taux d'accord inter-annotateurs constitue la limite haute 
              (\textbf{upper bound}) de ce qu'il est possible d'évaluer
    \end{itemize}

    \item de nombreuses mesures d'évaluation
    \begin{itemize}
        \item coefficient kappa de Cohen
        \item voir~\cite{artstein2008inter} pour plus de détails
    \end{itemize}

\end{itemize}

\end{frame}
%-E--------------------------------------------------------------------------E-%


%-B--------------------------------------------------------------------------B-%
\begin{frame}
\frametitle{Le découpage des données}

\begin{itemize} \itemsep0.8em
    \item la plupart des évaluations impliquent un découpage des données en 
          ensembles \textbf{disjoints}
    \begin{itemize}
        \item \textbf{training data} (70\%)
        \item \textbf{development data} (20\%)
        \item \textbf{test data} (10\%)
    \end{itemize}

    \item \textbf{évaluation croisée}
    \begin{itemize}
        \item permet d'évaluer sur toutes les données disponibles
        \item découper l'ensemble de données en $k$ partitions
        \begin{itemize}
            \item entraîner sur $k-1$ partitions et tester sur la partition restante
            \item calculer la performance moyenne sur les $k$ partitions
        \end{itemize}
    \end{itemize}
\end{itemize}


\end{frame}
%-E--------------------------------------------------------------------------E-%


%-B--------------------------------------------------------------------------B-%
\begin{frame}
\frametitle{Présenter des résultats}

\begin{itemize} \itemsep0.8em
    \item mesures/métriques pour estimer la performance
    \item toujours inclure au moins une \textit{baseline}
\end{itemize}

\vspace*{1em}

\small
\begin{tabular}{ l c c c }
\hline
    \textbf{Système} &
    \textbf{Mesure 1} & 
    \textbf{Mesure 2} & 
    \textbf{Mesure combinée} \\
\hline
    Baseline 1 & $M_1^{B1}$ & $M_2^{B1}$ & $M_c^{B1}$ \\
    Baseline 2 & $M_1^{B2}$ & $M_2^{B2}$ & $M_c^{B2}$ \\
    Variation 1 & $M_1^{V1}$ & $M_2^{V1}$ & $M_c^{V1}$ \\
    Variation 2 & $M_1^{V2}$ & $M_2^{V2}$ & $M_c^{V2}$ \\
    Upper bound & $M_1^U$ & $M_2^U$ & $M_c^U$ \\
\hline
\end{tabular}


\end{frame}
%-E--------------------------------------------------------------------------E-%

%-B--------------------------------------------------------------------------B-%
\begin{frame}
\frametitle{Test de significativité}

\begin{itemize} \itemsep0.8em
    \item considérons les performances (\textsc{Rouge-2}) de deux systèmes de résumé 
          automatique
    \begin{itemize}
        \item système 1~: 40\% vs. système 2~: 43\%
    \end{itemize}
\end{itemize}

\pause

\begin{center}

\begin{tabular}{ l | c c c c c || c }
\hline
~ & A & B & C & D & E & Avg. \\
\hline
    \textbf{système 1} & 
        \textcolor{ForestGreen}{\fbox{41}} & 
        \textcolor{ForestGreen}{\fbox{34}} & 
        \textcolor{ForestGreen}{\fbox{31}} & 
        35 & 
        \textcolor{ForestGreen}{\fbox{59}} & 40 \\
    \textbf{système 2} & 
        38 & 
        29 & 
        27 & 
        \textcolor{ForestGreen}{\fbox{65}} & 
        56 & 
        \textcolor{ForestGreen}{\fbox{43}} \\
\hline
\end{tabular}

\end{center}

\begin{itemize}
    \item toujours effectuer un test de significativité
    \begin{itemize}
        \item e.g.~ t.test de Student
    \end{itemize}
\end{itemize}

\end{frame}
%-E--------------------------------------------------------------------------E-%


% %-B--------------------------------------------------------------------------B-%
% \begin{frame}
% \frametitle{Introduction}


% \end{frame}
% %-E--------------------------------------------------------------------------E-%

%-E--------------------------------------------------------------------------E-%


%==============================================================================%
\section{Références}
%==============================================================================%

%-B--------------------------------------------------------------------------B-%
\begin{frame}[allowframebreaks]
    \fontsize{9pt}{10}\selectfont
    \frametitle{References}
    \bibliographystyle{alpha}
    \bibliography{bibliography}
\end{frame}
%-E--------------------------------------------------------------------------E-%


\end{document}