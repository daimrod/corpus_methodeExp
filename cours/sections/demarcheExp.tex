% Mise en place d'une démarche expérimentale : paramètres d'évaluation (quantité de données, découpage entraînement/dev/test, etc.), analyse des résultats, tests de significativité, tests de correlation

%-B--------------------------------------------------------------------------B-%
\begin{frame}[allowframebreaks]
\frametitle{Introduction}

\begin{itemize} \itemsep0.8em

    \item \textbf{démarche expérimentale} $\to$ valider une hypothèse par des 
          expériences
    \begin{itemize}
        \item e.g.~expérience du cerf-volant de Benjamin Franklin
    \end{itemize}

    \item exemple
    \begin{itemize}
        \item hypothèse $\to$ segmenter le chinois avec les CRF
        \item expérience $\to$ évaluer la précision sur un corpus
    \end{itemize}    

    \item \textbf{évaluer} un système de TAL consiste à vérifier qu'il produit 
          le résultat pour lequel il a été conçu

    \framebreak

    \item le TAL englobe un grand nombre de tâches, chacune ayant des critères 
          particuliers quant à son évaluation

    \item cette partie est basée sur~\cite{resnik2010}
    \begin{itemize}
        \item \url{http://www.umiacs.umd.edu/~jbg/teaching/CMSC_773_2012/reading/evaluation.pdf}
    \end{itemize}
\end{itemize}

\end{frame}
%-E--------------------------------------------------------------------------E-%

%-B--------------------------------------------------------------------------B-%
\begin{frame}[allowframebreaks]
\frametitle{Exemple}

\begin{itemize} \itemsep0.8em
    \item un exemple pour introduire les idées que nous allons voir en détails 
          par la suite

    \item évaluer un moteur de recherche sémantique
    \begin{itemize}
        \item requêtes analysées et converties en sujet-relation-objet
    \end{itemize}

    \item \textit{when was the light bulb patented by Edison~?}
    \begin{itemize}
        \item[$\to$] $[$Edison, patented, bulb$]$
        \item permet de retrouver le document \og{}\textit{Thomas Edison's 
              patent of the electric light bulb}\fg{}
    \end{itemize}

    \framebreak

    \item comment évaluer le composant d'analyse en sujet-relation-objet~?

    \item[$\to$] conduire une évaluation \textbf{intrinsèque}
    \begin{itemize}
        \item créer un ensemble de questions analysées manuellement
        \item évaluer la performance avec des mesures de P-R-F
         \item comparer deux versions (\textit{formative evaluation})
        \item comparer à d'autres méthodes (\textit{summative evaluation})
    \end{itemize}

    \framebreak

    \item est-ce qu'une analyse précise permet d'améliorer les résultats du 
          moteur de recherche~?

    \item[$\to$] conduire une évaluation \textbf{extrinsèque}
    \begin{itemize}
        \item évaluer l'impact de l'analyse sur la performance du moteur
        \item utiliser une collection de test avec des mesures de P-R-F
    \end{itemize}

    \item[$\to$] conduire évaluation \textit{in situ}
    \begin{itemize}
        \item proposer le système aux utilisateurs et les observer
    \end{itemize}

\end{itemize}

\end{frame}
%-E--------------------------------------------------------------------------E-%

%-B--------------------------------------------------------------------------B-%
\begin{frame}[allowframebreaks]
\frametitle{Concepts fondamentaux}

\begin{itemize} \itemsep0.8em
    \item \textbf{évaluation manuelle}
    \begin{itemize}
        \item demander à des sujets humains d'évaluer un système selon des 
              critères pré-définis $\to$ souvent la meilleure évaluation~!
        \item nombreuses limitations~: coût important, évaluation lente, 
              résultats inconsistants et non reproductibles
    \end{itemize}

    \item \textbf{évaluation automatique}
    \begin{itemize}
        \item création d'un \textit{gold standard}, \textit{ground truth}, etc.
        \item nécessite une mesure qui \og{}simule\fg{} une évaluation manuelle
        \item correlation entre les évaluations manuelles et automatiques
    \end{itemize}

    \framebreak

    \item \textbf{évaluation intrinsèque}
    \begin{itemize}
        \item la sortie du système est évaluée directement par rapport à des 
              critères pré-définis
    \end{itemize}

    \item \textbf{évaluation extrinsèque}
    \begin{itemize}
        \item la sortie du système est évaluée à travers son impact sur une 
              tâche externe
    \end{itemize}

    \item exemple du résumé automatique
    \begin{itemize}
        \item évaluation intrinsèque $\to$ ROUGE ou évaluation manuelle
        \item évaluation extrinsèque $\to$ les résumés sont-ils utiles pour 
              remplacer les snippets d'un moteur de RI~?
    \end{itemize}

    \framebreak

    \item \textbf{accord inter-annotateurs}
    \begin{itemize}
        \item en TAL, l'évaluation se résume à annoter du texte
        \item comparer la performance de plusieurs annotateurs
        \item un accord inter-annotateurs faible
        \begin{itemize}
            \item[$\to$] tâche trop difficile ou mal définie
        \end{itemize}
        \item le taux d'accord inter-annotateurs constitue la limite haute 
              (\textbf{upper bound}) de ce qu'il est possible d'évaluer
    \end{itemize}

    \item de nombreuses mesures d'évaluation
    \begin{itemize}
        \item coefficient kappa de Cohen
        \item voir~\cite{artstein2008inter} pour plus de détails
    \end{itemize}

\end{itemize}

\end{frame}
%-E--------------------------------------------------------------------------E-%


%-B--------------------------------------------------------------------------B-%
\begin{frame}
\frametitle{Le découpage des données}

\begin{itemize} \itemsep0.8em
    \item la plupart des évaluations impliquent un découpage des données en 
          ensembles \textbf{disjoints}
    \begin{itemize}
        \item \textbf{training data} (70\%)
        \item \textbf{development data} (20\%)
        \item \textbf{test data} (10\%)
    \end{itemize}

    \item \textbf{évaluation croisée}
    \begin{itemize}
        \item permet d'évaluer sur toutes les données disponibles
        \item découper l'ensemble de données en $k$ partitions
        \begin{itemize}
            \item entraîner sur $k-1$ partitions et tester sur la partition restante
            \item calculer la performance moyenne sur les $k$ partitions
        \end{itemize}
    \end{itemize}
\end{itemize}


\end{frame}
%-E--------------------------------------------------------------------------E-%


%-B--------------------------------------------------------------------------B-%
\begin{frame}
\frametitle{Présenter des résultats}

\begin{itemize} \itemsep0.8em
    \item mesures/métriques pour estimer la performance
    \item toujours inclure au moins une \textit{baseline}
\end{itemize}

\vspace*{1em}

\small
\begin{tabular}{ l c c c }
\hline
    \textbf{Système} &
    \textbf{Mesure 1} & 
    \textbf{Mesure 2} & 
    \textbf{Mesure combinée} \\
\hline
    Baseline 1 & $M_1^{B1}$ & $M_2^{B1}$ & $M_c^{B1}$ \\
    Baseline 2 & $M_1^{B2}$ & $M_2^{B2}$ & $M_c^{B2}$ \\
    Variation 1 & $M_1^{V1}$ & $M_2^{V1}$ & $M_c^{V1}$ \\
    Variation 2 & $M_1^{V2}$ & $M_2^{V2}$ & $M_c^{V2}$ \\
    Upper bound & $M_1^U$ & $M_2^U$ & $M_c^U$ \\
\hline
\end{tabular}


\end{frame}
%-E--------------------------------------------------------------------------E-%

%-B--------------------------------------------------------------------------B-%
\begin{frame}
\frametitle{Test de significativité}

\begin{itemize} \itemsep0.8em
    \item considérons les performances (\textsc{Rouge-2}) de deux systèmes de résumé 
          automatique
    \begin{itemize}
        \item système 1~: 40\% vs. système 2~: 43\%
    \end{itemize}
\end{itemize}

\pause

\begin{center}

\begin{tabular}{ l | c c c c c || c }
\hline
~ & A & B & C & D & E & Avg. \\
\hline
    \textbf{système 1} & 
        \textcolor{ForestGreen}{\fbox{41}} & 
        \textcolor{ForestGreen}{\fbox{34}} & 
        \textcolor{ForestGreen}{\fbox{31}} & 
        35 & 
        \textcolor{ForestGreen}{\fbox{59}} & 40 \\
    \textbf{système 2} & 
        38 & 
        29 & 
        27 & 
        \textcolor{ForestGreen}{\fbox{65}} & 
        56 & 
        \textcolor{ForestGreen}{\fbox{43}} \\
\hline
\end{tabular}

\end{center}

\begin{itemize}
    \item toujours effectuer un test de significativité
    \begin{itemize}
        \item e.g.~ t.test de Student
    \end{itemize}
\end{itemize}

\end{frame}
%-E--------------------------------------------------------------------------E-%


% %-B--------------------------------------------------------------------------B-%
% \begin{frame}
% \frametitle{Introduction}


% \end{frame}
% %-E--------------------------------------------------------------------------E-%
