% Mise en place d'une démarche expérimentale : paramètres d'évaluation (quantité de données, découpage entraînement/dev/test, etc.), analyse des résultats, tests de significativité, tests de correlation

%-B--------------------------------------------------------------------------B-%
\begin{frame}
\frametitle{Introduction}

\begin{itemize} \itemsep0.8em

    \item \textbf{démarche expérimentale} $\to$ valider une hypothèse par des 
          expériences
    \begin{itemize}
        \item e.g.~expérience du cerf-volant de Benjamin Franklin
    \end{itemize}
    \item \textbf{évaluer} un système de TAL consiste à vérifier qu'il produit 
          le résultat pour lequel il a été conçu

    \item le TAL regroupe 
\end{itemize}



\end{frame}
%-E--------------------------------------------------------------------------E-%


%-B--------------------------------------------------------------------------B-%
\begin{frame}
\frametitle{Introduction}


\end{frame}
%-E--------------------------------------------------------------------------E-%