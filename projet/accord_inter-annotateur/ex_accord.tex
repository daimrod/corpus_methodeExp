\documentclass[12pt,a4paper]{article}
\usepackage[top=1cm, bottom=1.5cm, left=1cm, right=1cm]{geometry}
\usepackage[utf8]{inputenc}
\usepackage[french]{babel}
\usepackage[T1]{fontenc}
\usepackage{url}

% \setlength{\parskip}{0.8em}

\begin{document}

\title{Notation des présentations et accord inter-annotateur}
\author{Florian Boudin}
\date{Corpus et méthode expérimentale - 2013}

\maketitle

\section*{Notation des présentations}

Vous allez présenter, par groupe de deux étudiants, quatre articles de recherche.
Cette présentation est notée et compte pour 25\% de la note de ce module.
Vous serez évalués sur cinq critères~:

\begin{enumerate}
    \item[C1] Qualité des transparents (1-5)
    \item[C2] Qualité du discours et répartition entre les deux étudiants (1-5)
    \item[C3] Compréhension et vulgarisation du contenu scientifique (1-5)
    \item[C4] Réponse aux questions (1-5)
    \item[C5] Respect du temps (1-5)
    \item[C6] Qualité générale de la présentation (1-5)
\end{enumerate}

En plus des notes que je vais vous donner, vous effecturez également une évaluation de chaque présentation.
L'intérêt de cette évaluation multiple est double~: 1.~vous faire participer à la notation, et 2.~recueillir des évaluations pour calculer des mesures d'accord inter-annotateur.
Un formulaire de notation vous sera remis en début de séance.

\section*{Calcul de l'accord inter-annotateur}

Vos évaluations ont été collectées, \textbf{anonymisées} et regroupées dans un fichier au format suivant~:

\begin{verbatim}
id_annotateur \t C1 \t C2 \t C3 \t C4 \t C5 \t C6
\end{verbatim}

Dans un premier temps, calculez des indices statistiques permettant de répondre aux questions suivantes~:

\begin{enumerate}
    \item Quelle présentation a obtenu les meilleures notes~?
    \item Quel est l'annotateur le plus strict/le plus généreux~?
    \item Les différences de notes entre les présentations sont-elles significatives~?
    \item Quel est le critère d'annotation le plus subjectif~?
\end{enumerate}


Votre tâche consiste ensuite à calculer l'accord inter-annotateur (\textit{Fleiss kappa}) pour chaque critère, une description de cette mesure est disponible sur wikipédia\footnote{\url{http://en.wikipedia.org/wiki/Fleiss_kappa}}.







\end{document}  