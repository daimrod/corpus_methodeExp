\documentclass[12pt,a4paper]{article}
\usepackage[top=1cm, bottom=1.5cm, left=1cm, right=1cm]{geometry}
\usepackage[utf8]{inputenc}
\usepackage[french]{babel}
\usepackage[T1]{fontenc}
\usepackage{url}

\begin{document}

\title{Création de corpus~: contextes de citations}
\author{Florian Boudin}
\date{Corpus et méthode expérimentale - 2013}

\maketitle

\section*{Méthodologie à suivre}

\begin{enumerate}
    \item Lecture d'articles
    \begin{itemize}
        \item Radev et Abu-Jbara, \textit{Rediscovering ACL discoveries through the lens of ACL anthology network citing sentences}, \url{http://www.aclweb.org/anthology-new/W/W12/W12-3201.pdf}.
        \item Teufel, Siddharthan et Tidhar, \textit{Automatic classification of citation function}, \url{http://www.aclweb.org/anthology-new/W/W06/W06-1613.pdf}.
        \item Qazvinian et Radev, \textit{Identifying Non-Explicit Citing Sentences for Citation-Based Summarization}, \url{http://www.aclweb.org/anthology-new/P/P10/P10-1057.pdf}.
        \item Elkiss, Shen, Fader, Erkan, States et Radev. \textit{Blind men and elephants : What do citation summaries tell us about a research article?}, \url{http://homes.cs.washington.edu/~afader/bib_pdf/jasist08.pdf}
    \end{itemize}                                                                                                                                 
    \item Etat de l'art des corpus existants
    \begin{itemize}
        \item e.g.~\textit{ACL Anthology Network}
    \end{itemize}
    \item Définir les caractéristiques du corpus
    \begin{itemize}
        \item Nature des documents, droits de redistribution, taille, etc.
    \end{itemize}
    \item Annotation du corpus
    \begin{itemize}
        \item Définition des \textit{guidelines} puis test et modification
    \end{itemize}
    \item Valorisation du corpus créé
    \begin{itemize}
        \item e.g.~article soumis à LREC, RECITAL
    \end{itemize}

\end{enumerate}





\end{document}  